% Generated by Sphinx.
\def\sphinxdocclass{report}
\documentclass[letterpaper,10pt,english]{sphinxmanual}
\usepackage[utf8]{inputenc}
\DeclareUnicodeCharacter{00A0}{\nobreakspace}
\usepackage{cmap}
\usepackage[T1]{fontenc}
\usepackage{babel}
\usepackage{times}
\usepackage[Bjarne]{fncychap}
\usepackage{longtable}
\usepackage{sphinx}
\usepackage{multirow}

\addto\captionsenglish{\renewcommand{\figurename}{Fig. }}
\addto\captionsenglish{\renewcommand{\tablename}{Table }}
\floatname{literal-block}{Listing }



\title{pymzavro Documentation}
\date{August 15, 2015}
\release{0.5}
\author{Marius}
\newcommand{\sphinxlogo}{}
\renewcommand{\releasename}{Release}
\makeindex

\makeatletter
\def\PYG@reset{\let\PYG@it=\relax \let\PYG@bf=\relax%
    \let\PYG@ul=\relax \let\PYG@tc=\relax%
    \let\PYG@bc=\relax \let\PYG@ff=\relax}
\def\PYG@tok#1{\csname PYG@tok@#1\endcsname}
\def\PYG@toks#1+{\ifx\relax#1\empty\else%
    \PYG@tok{#1}\expandafter\PYG@toks\fi}
\def\PYG@do#1{\PYG@bc{\PYG@tc{\PYG@ul{%
    \PYG@it{\PYG@bf{\PYG@ff{#1}}}}}}}
\def\PYG#1#2{\PYG@reset\PYG@toks#1+\relax+\PYG@do{#2}}

\expandafter\def\csname PYG@tok@gd\endcsname{\def\PYG@tc##1{\textcolor[rgb]{0.63,0.00,0.00}{##1}}}
\expandafter\def\csname PYG@tok@gu\endcsname{\let\PYG@bf=\textbf\def\PYG@tc##1{\textcolor[rgb]{0.50,0.00,0.50}{##1}}}
\expandafter\def\csname PYG@tok@gt\endcsname{\def\PYG@tc##1{\textcolor[rgb]{0.00,0.27,0.87}{##1}}}
\expandafter\def\csname PYG@tok@gs\endcsname{\let\PYG@bf=\textbf}
\expandafter\def\csname PYG@tok@gr\endcsname{\def\PYG@tc##1{\textcolor[rgb]{1.00,0.00,0.00}{##1}}}
\expandafter\def\csname PYG@tok@cm\endcsname{\let\PYG@it=\textit\def\PYG@tc##1{\textcolor[rgb]{0.25,0.50,0.56}{##1}}}
\expandafter\def\csname PYG@tok@vg\endcsname{\def\PYG@tc##1{\textcolor[rgb]{0.73,0.38,0.84}{##1}}}
\expandafter\def\csname PYG@tok@m\endcsname{\def\PYG@tc##1{\textcolor[rgb]{0.13,0.50,0.31}{##1}}}
\expandafter\def\csname PYG@tok@mh\endcsname{\def\PYG@tc##1{\textcolor[rgb]{0.13,0.50,0.31}{##1}}}
\expandafter\def\csname PYG@tok@cs\endcsname{\def\PYG@tc##1{\textcolor[rgb]{0.25,0.50,0.56}{##1}}\def\PYG@bc##1{\setlength{\fboxsep}{0pt}\colorbox[rgb]{1.00,0.94,0.94}{\strut ##1}}}
\expandafter\def\csname PYG@tok@ge\endcsname{\let\PYG@it=\textit}
\expandafter\def\csname PYG@tok@vc\endcsname{\def\PYG@tc##1{\textcolor[rgb]{0.73,0.38,0.84}{##1}}}
\expandafter\def\csname PYG@tok@il\endcsname{\def\PYG@tc##1{\textcolor[rgb]{0.13,0.50,0.31}{##1}}}
\expandafter\def\csname PYG@tok@go\endcsname{\def\PYG@tc##1{\textcolor[rgb]{0.20,0.20,0.20}{##1}}}
\expandafter\def\csname PYG@tok@cp\endcsname{\def\PYG@tc##1{\textcolor[rgb]{0.00,0.44,0.13}{##1}}}
\expandafter\def\csname PYG@tok@gi\endcsname{\def\PYG@tc##1{\textcolor[rgb]{0.00,0.63,0.00}{##1}}}
\expandafter\def\csname PYG@tok@gh\endcsname{\let\PYG@bf=\textbf\def\PYG@tc##1{\textcolor[rgb]{0.00,0.00,0.50}{##1}}}
\expandafter\def\csname PYG@tok@ni\endcsname{\let\PYG@bf=\textbf\def\PYG@tc##1{\textcolor[rgb]{0.84,0.33,0.22}{##1}}}
\expandafter\def\csname PYG@tok@nl\endcsname{\let\PYG@bf=\textbf\def\PYG@tc##1{\textcolor[rgb]{0.00,0.13,0.44}{##1}}}
\expandafter\def\csname PYG@tok@nn\endcsname{\let\PYG@bf=\textbf\def\PYG@tc##1{\textcolor[rgb]{0.05,0.52,0.71}{##1}}}
\expandafter\def\csname PYG@tok@no\endcsname{\def\PYG@tc##1{\textcolor[rgb]{0.38,0.68,0.84}{##1}}}
\expandafter\def\csname PYG@tok@na\endcsname{\def\PYG@tc##1{\textcolor[rgb]{0.25,0.44,0.63}{##1}}}
\expandafter\def\csname PYG@tok@nb\endcsname{\def\PYG@tc##1{\textcolor[rgb]{0.00,0.44,0.13}{##1}}}
\expandafter\def\csname PYG@tok@nc\endcsname{\let\PYG@bf=\textbf\def\PYG@tc##1{\textcolor[rgb]{0.05,0.52,0.71}{##1}}}
\expandafter\def\csname PYG@tok@nd\endcsname{\let\PYG@bf=\textbf\def\PYG@tc##1{\textcolor[rgb]{0.33,0.33,0.33}{##1}}}
\expandafter\def\csname PYG@tok@ne\endcsname{\def\PYG@tc##1{\textcolor[rgb]{0.00,0.44,0.13}{##1}}}
\expandafter\def\csname PYG@tok@nf\endcsname{\def\PYG@tc##1{\textcolor[rgb]{0.02,0.16,0.49}{##1}}}
\expandafter\def\csname PYG@tok@si\endcsname{\let\PYG@it=\textit\def\PYG@tc##1{\textcolor[rgb]{0.44,0.63,0.82}{##1}}}
\expandafter\def\csname PYG@tok@s2\endcsname{\def\PYG@tc##1{\textcolor[rgb]{0.25,0.44,0.63}{##1}}}
\expandafter\def\csname PYG@tok@vi\endcsname{\def\PYG@tc##1{\textcolor[rgb]{0.73,0.38,0.84}{##1}}}
\expandafter\def\csname PYG@tok@nt\endcsname{\let\PYG@bf=\textbf\def\PYG@tc##1{\textcolor[rgb]{0.02,0.16,0.45}{##1}}}
\expandafter\def\csname PYG@tok@nv\endcsname{\def\PYG@tc##1{\textcolor[rgb]{0.73,0.38,0.84}{##1}}}
\expandafter\def\csname PYG@tok@s1\endcsname{\def\PYG@tc##1{\textcolor[rgb]{0.25,0.44,0.63}{##1}}}
\expandafter\def\csname PYG@tok@gp\endcsname{\let\PYG@bf=\textbf\def\PYG@tc##1{\textcolor[rgb]{0.78,0.36,0.04}{##1}}}
\expandafter\def\csname PYG@tok@sh\endcsname{\def\PYG@tc##1{\textcolor[rgb]{0.25,0.44,0.63}{##1}}}
\expandafter\def\csname PYG@tok@ow\endcsname{\let\PYG@bf=\textbf\def\PYG@tc##1{\textcolor[rgb]{0.00,0.44,0.13}{##1}}}
\expandafter\def\csname PYG@tok@sx\endcsname{\def\PYG@tc##1{\textcolor[rgb]{0.78,0.36,0.04}{##1}}}
\expandafter\def\csname PYG@tok@bp\endcsname{\def\PYG@tc##1{\textcolor[rgb]{0.00,0.44,0.13}{##1}}}
\expandafter\def\csname PYG@tok@c1\endcsname{\let\PYG@it=\textit\def\PYG@tc##1{\textcolor[rgb]{0.25,0.50,0.56}{##1}}}
\expandafter\def\csname PYG@tok@kc\endcsname{\let\PYG@bf=\textbf\def\PYG@tc##1{\textcolor[rgb]{0.00,0.44,0.13}{##1}}}
\expandafter\def\csname PYG@tok@c\endcsname{\let\PYG@it=\textit\def\PYG@tc##1{\textcolor[rgb]{0.25,0.50,0.56}{##1}}}
\expandafter\def\csname PYG@tok@mf\endcsname{\def\PYG@tc##1{\textcolor[rgb]{0.13,0.50,0.31}{##1}}}
\expandafter\def\csname PYG@tok@err\endcsname{\def\PYG@bc##1{\setlength{\fboxsep}{0pt}\fcolorbox[rgb]{1.00,0.00,0.00}{1,1,1}{\strut ##1}}}
\expandafter\def\csname PYG@tok@mb\endcsname{\def\PYG@tc##1{\textcolor[rgb]{0.13,0.50,0.31}{##1}}}
\expandafter\def\csname PYG@tok@ss\endcsname{\def\PYG@tc##1{\textcolor[rgb]{0.32,0.47,0.09}{##1}}}
\expandafter\def\csname PYG@tok@sr\endcsname{\def\PYG@tc##1{\textcolor[rgb]{0.14,0.33,0.53}{##1}}}
\expandafter\def\csname PYG@tok@mo\endcsname{\def\PYG@tc##1{\textcolor[rgb]{0.13,0.50,0.31}{##1}}}
\expandafter\def\csname PYG@tok@kd\endcsname{\let\PYG@bf=\textbf\def\PYG@tc##1{\textcolor[rgb]{0.00,0.44,0.13}{##1}}}
\expandafter\def\csname PYG@tok@mi\endcsname{\def\PYG@tc##1{\textcolor[rgb]{0.13,0.50,0.31}{##1}}}
\expandafter\def\csname PYG@tok@kn\endcsname{\let\PYG@bf=\textbf\def\PYG@tc##1{\textcolor[rgb]{0.00,0.44,0.13}{##1}}}
\expandafter\def\csname PYG@tok@o\endcsname{\def\PYG@tc##1{\textcolor[rgb]{0.40,0.40,0.40}{##1}}}
\expandafter\def\csname PYG@tok@kr\endcsname{\let\PYG@bf=\textbf\def\PYG@tc##1{\textcolor[rgb]{0.00,0.44,0.13}{##1}}}
\expandafter\def\csname PYG@tok@s\endcsname{\def\PYG@tc##1{\textcolor[rgb]{0.25,0.44,0.63}{##1}}}
\expandafter\def\csname PYG@tok@kp\endcsname{\def\PYG@tc##1{\textcolor[rgb]{0.00,0.44,0.13}{##1}}}
\expandafter\def\csname PYG@tok@w\endcsname{\def\PYG@tc##1{\textcolor[rgb]{0.73,0.73,0.73}{##1}}}
\expandafter\def\csname PYG@tok@kt\endcsname{\def\PYG@tc##1{\textcolor[rgb]{0.56,0.13,0.00}{##1}}}
\expandafter\def\csname PYG@tok@sc\endcsname{\def\PYG@tc##1{\textcolor[rgb]{0.25,0.44,0.63}{##1}}}
\expandafter\def\csname PYG@tok@sb\endcsname{\def\PYG@tc##1{\textcolor[rgb]{0.25,0.44,0.63}{##1}}}
\expandafter\def\csname PYG@tok@k\endcsname{\let\PYG@bf=\textbf\def\PYG@tc##1{\textcolor[rgb]{0.00,0.44,0.13}{##1}}}
\expandafter\def\csname PYG@tok@se\endcsname{\let\PYG@bf=\textbf\def\PYG@tc##1{\textcolor[rgb]{0.25,0.44,0.63}{##1}}}
\expandafter\def\csname PYG@tok@sd\endcsname{\let\PYG@it=\textit\def\PYG@tc##1{\textcolor[rgb]{0.25,0.44,0.63}{##1}}}

\def\PYGZbs{\char`\\}
\def\PYGZus{\char`\_}
\def\PYGZob{\char`\{}
\def\PYGZcb{\char`\}}
\def\PYGZca{\char`\^}
\def\PYGZam{\char`\&}
\def\PYGZlt{\char`\<}
\def\PYGZgt{\char`\>}
\def\PYGZsh{\char`\#}
\def\PYGZpc{\char`\%}
\def\PYGZdl{\char`\$}
\def\PYGZhy{\char`\-}
\def\PYGZsq{\char`\'}
\def\PYGZdq{\char`\"}
\def\PYGZti{\char`\~}
% for compatibility with earlier versions
\def\PYGZat{@}
\def\PYGZlb{[}
\def\PYGZrb{]}
\makeatother

\renewcommand\PYGZsq{\textquotesingle}

\begin{document}

\maketitle
\tableofcontents
\phantomsection\label{index::doc}


This is the documentation of pymzavro
\begin{description}
\item[{Requirements:}] \leavevmode
-None

\end{description}

Contents:


\chapter{This is a short Tutorial of how to use pymzavro}
\label{tutorial:welcome-to-pymzavro-s-documentation}\label{tutorial:this-is-a-short-tutorial-of-how-to-use-pymzavro}\label{tutorial::doc}

\chapter{mzMLWriter}
\label{mzMLWriter:mzmlwriter}\label{mzMLWriter::doc}\label{mzMLWriter:module-mzMLWriter}\index{mzMLWriter (module)}\index{mzMLWriter (class in mzMLWriter)}

\begin{fulllineitems}
\phantomsection\label{mzMLWriter:mzMLWriter.mzMLWriter}\pysigline{\strong{class }\code{mzMLWriter.}\bfcode{mzMLWriter}}
Class to convert mzML to avro. Uses pyavroc and pymzML. It is possible that the size of a single record could
cause problems with the default buffer size of pyavroc. Its recommened to patch the buffer size in datafile.c.

Example:
\begin{description}
\item[{Import pymzavro and pymzml (to iterate)}] \leavevmode
\begin{Verbatim}[commandchars=\\\{\}]
\PYG{g+gp}{\PYGZgt{}\PYGZgt{}\PYGZgt{} }\PYG{k+kn}{import} \PYG{n+nn}{pymzavro}
\PYG{g+gp}{\PYGZgt{}\PYGZgt{}\PYGZgt{} }\PYG{k+kn}{import} \PYG{n+nn}{pymzml}
\end{Verbatim}

\item[{Open required files}] \leavevmode
\begin{Verbatim}[commandchars=\\\{\}]
\PYG{g+gp}{\PYGZgt{}\PYGZgt{}\PYGZgt{} }\PYG{n}{mzML} \PYG{o}{=} \PYG{n+nb}{open}\PYG{p}{(}\PYG{l+s}{\PYGZdq{}}\PYG{l+s}{/home/marius/data/F/mzML/F04.mzML}\PYG{l+s}{\PYGZdq{}}\PYG{p}{,} \PYG{l+s}{\PYGZdq{}}\PYG{l+s}{r}\PYG{l+s}{\PYGZdq{}}\PYG{p}{)}
\PYG{g+gp}{\PYGZgt{}\PYGZgt{}\PYGZgt{} }\PYG{n}{spectrumFile} \PYG{o}{=} \PYG{n+nb}{open}\PYG{p}{(}\PYG{l+s}{\PYGZdq{}}\PYG{l+s}{F04\PYGZus{}deflate.avro}\PYG{l+s}{\PYGZdq{}}\PYG{p}{,} \PYG{l+s}{\PYGZdq{}}\PYG{l+s}{wb}\PYG{l+s}{\PYGZdq{}}\PYG{p}{)}
\PYG{g+gp}{\PYGZgt{}\PYGZgt{}\PYGZgt{} }\PYG{n}{spectrumSchema} \PYG{o}{=} \PYG{n+nb}{open}\PYG{p}{(}\PYG{l+s}{\PYGZdq{}}\PYG{l+s}{spectrum.avsc}\PYG{l+s}{\PYGZdq{}}\PYG{p}{,}\PYG{l+s}{\PYGZdq{}}\PYG{l+s}{r}\PYG{l+s}{\PYGZdq{}}\PYG{p}{)}
\PYG{g+gp}{\PYGZgt{}\PYGZgt{}\PYGZgt{} }\PYG{n}{metaDataFile} \PYG{o}{=} \PYG{n+nb}{open}\PYG{p}{(}\PYG{l+s}{\PYGZdq{}}\PYG{l+s}{F04meta\PYGZus{}deflate.avro}\PYG{l+s}{\PYGZdq{}}\PYG{p}{,} \PYG{l+s}{\PYGZdq{}}\PYG{l+s}{wb}\PYG{l+s}{\PYGZdq{}}\PYG{p}{)}
\PYG{g+gp}{\PYGZgt{}\PYGZgt{}\PYGZgt{} }\PYG{n}{typeDict} \PYG{o}{=} \PYG{n+nb}{open}\PYG{p}{(}\PYG{l+s}{\PYGZdq{}}\PYG{l+s}{typeDict.json}\PYG{l+s}{\PYGZdq{}}\PYG{p}{,} \PYG{l+s}{\PYGZdq{}}\PYG{l+s}{rb}\PYG{l+s}{\PYGZdq{}}\PYG{p}{)}
\PYG{g+gp}{\PYGZgt{}\PYGZgt{}\PYGZgt{} }\PYG{n}{metaDataSchema} \PYG{o}{=} \PYG{n+nb}{open}\PYG{p}{(}\PYG{l+s}{\PYGZdq{}}\PYG{l+s}{fullSchema.avsc}\PYG{l+s}{\PYGZdq{}}\PYG{p}{,} \PYG{l+s}{\PYGZdq{}}\PYG{l+s}{r}\PYG{l+s}{\PYGZdq{}}\PYG{p}{)}
\PYG{g+gp}{\PYGZgt{}\PYGZgt{}\PYGZgt{} }\PYG{n}{indexFile} \PYG{o}{=} \PYG{l+s}{\PYGZdq{}}\PYG{l+s}{F04\PYGZus{}Indexdeflate.json}\PYG{l+s}{\PYGZdq{}}
\end{Verbatim}

\item[{Create writer object and initialize it correctly}] \leavevmode
\begin{Verbatim}[commandchars=\\\{\}]
\PYG{g+gp}{\PYGZgt{}\PYGZgt{}\PYGZgt{} }\PYG{n}{writer} \PYG{o}{=} \PYG{n}{pymzavro}\PYG{o}{.}\PYG{n}{mzMLWriter}\PYG{o}{.}\PYG{n}{mzMLWriter}\PYG{p}{(}\PYG{p}{)}
\end{Verbatim}

\begin{Verbatim}[commandchars=\\\{\}]
\PYG{g+gp}{\PYGZgt{}\PYGZgt{}\PYGZgt{} }\PYG{n}{writer}\PYG{o}{.}\PYG{n}{init\PYGZus{}file}\PYG{p}{(}\PYG{n}{mzML}\PYG{p}{,} \PYG{n}{spectrumFile}\PYG{p}{,} \PYG{n}{typeDict}\PYG{o}{=}\PYG{n}{typeDict}\PYG{p}{,} \PYG{n}{spectrumAvsc}\PYG{o}{=}\PYG{n}{spectrumSchema}\PYG{p}{,}            \PYG{n}{avro\PYGZus{}schema}\PYG{o}{=}\PYG{n}{metaDataSchema}\PYG{p}{,} \PYG{n}{metaDataFile}\PYG{o}{=}\PYG{n}{metaDataFile}\PYG{p}{,} \PYG{n}{indexJSON}\PYG{o}{=}\PYG{n}{indexFile}
\PYG{g+gp}{\PYGZgt{}\PYGZgt{}\PYGZgt{} }\PYG{n}{writer}\PYG{o}{.}\PYG{n}{initmzAvroWriter}\PYG{p}{(}\PYG{p}{)}
\end{Verbatim}

\item[{Initialize reader (pymzML can be replaced by any other reader that offers the needed informations}] \leavevmode
\begin{Verbatim}[commandchars=\\\{\}]
\PYG{g+gp}{\PYGZgt{}\PYGZgt{}\PYGZgt{} }\PYG{n}{reader} \PYG{o}{=} \PYG{n}{pymzml}\PYG{o}{.}\PYG{n}{run}\PYG{o}{.}\PYG{n}{Reader}\PYG{p}{(}\PYG{l+s}{\PYGZdq{}}\PYG{l+s}{BSA3.avro}\PYG{l+s}{\PYGZdq{}}\PYG{p}{)}
\end{Verbatim}

\item[{Write metadata (optional), can be used for simple writer as well}] \leavevmode
\begin{Verbatim}[commandchars=\\\{\}]
\PYG{g+gp}{\PYGZgt{}\PYGZgt{}\PYGZgt{} }\PYG{n}{writer}\PYG{o}{.}\PYG{n}{writemzAvroMeta}\PYG{p}{(}\PYG{p}{)}
\end{Verbatim}

\end{description}

Iterator across the spectra and
get original XML Tree and decoded m/z and intensity array, arrays are added to a dict (see schema maker):

\begin{Verbatim}[commandchars=\\\{\}]
\PYG{g+gp}{\PYGZgt{}\PYGZgt{}\PYGZgt{} }\PYG{k}{for} \PYG{n}{spectrum} \PYG{o+ow}{in} \PYG{n}{reader}\PYG{p}{:}
\PYG{g+gp}{\PYGZgt{}\PYGZgt{}\PYGZgt{} }    \PYG{n}{xmlSpectrum} \PYG{o}{=} \PYG{n}{spectrum}\PYG{o}{.}\PYG{n}{xmlTreeIterFree}
\PYG{g+gp}{\PYGZgt{}\PYGZgt{}\PYGZgt{} }    \PYG{n}{mzArray} \PYG{o}{=} \PYG{n+nb}{list}\PYG{p}{(}\PYG{n}{spectrum}\PYG{o}{.}\PYG{n}{mz}\PYG{p}{)}
\PYG{g+gp}{\PYGZgt{}\PYGZgt{}\PYGZgt{} }    \PYG{n}{iArray} \PYG{o}{=} \PYG{n+nb}{list}\PYG{p}{(}\PYG{n}{spectrum}\PYG{o}{.}\PYG{n}{i}\PYG{p}{)}
\PYG{g+gp}{\PYGZgt{}\PYGZgt{}\PYGZgt{} }    \PYG{n}{dataDict} \PYG{o}{=} \PYG{p}{\PYGZob{}}\PYG{l+s}{\PYGZdq{}}\PYG{l+s}{mzArray}\PYG{l+s}{\PYGZdq{}} \PYG{p}{:} \PYG{n}{mzArray}\PYG{p}{,} \PYG{l+s}{\PYGZdq{}}\PYG{l+s}{intensityArray}\PYG{l+s}{\PYGZdq{}} \PYG{p}{:} \PYG{n}{iArray}\PYG{p}{\PYGZcb{}}
\end{Verbatim}
\begin{description}
\item[{Write data to avro}] \leavevmode
\begin{Verbatim}[commandchars=\\\{\}]
\PYG{g+gp}{\PYGZgt{}\PYGZgt{}\PYGZgt{} }    \PYG{n}{writer}\PYG{o}{.}\PYG{n}{mzAvroWriter}\PYG{p}{(}\PYG{n}{xmlSpectrum}\PYG{p}{,} \PYG{n}{dataDict}\PYG{p}{)}
\end{Verbatim}

\item[{Write index}] \leavevmode
\begin{Verbatim}[commandchars=\\\{\}]
\PYG{g+gp}{\PYGZgt{}\PYGZgt{}\PYGZgt{} }\PYG{n}{writer}\PYG{o}{.}\PYG{n}{writeOffsetToJson}\PYG{p}{(}\PYG{p}{)}
\end{Verbatim}

\end{description}
\index{advSeek() (mzMLWriter.mzMLWriter method)}

\begin{fulllineitems}
\phantomsection\label{mzMLWriter:mzMLWriter.mzMLWriter.advSeek}\pysiglinewithargsret{\bfcode{advSeek}}{\emph{index}}{}
This method is used to create an index based on the written blocks. To access a spectrum by index, seek() is
used to jump to the start of the block and the an iterator goes across the spectra in that block to find the one
with the given index.
\begin{quote}
\begin{quote}\begin{description}
\item[{param index}] \leavevmode
The index of the spectrum

\end{description}\end{quote}
\end{quote}

\end{fulllineitems}

\index{init\_file() (mzMLWriter.mzMLWriter method)}

\begin{fulllineitems}
\phantomsection\label{mzMLWriter:mzMLWriter.mzMLWriter.init_file}\pysiglinewithargsret{\bfcode{init\_file}}{\emph{mzML}, \emph{avroFile}, \emph{metaDataFile=None}, \emph{typeDict=None}, \emph{avro\_schema=None}, \emph{spectrumAvsc=None}, \emph{indexJSON='index.json'}}{}
The init file method is used to provide information about the files the Writer should use. All files should be
opened file like objects, except for the indexJSON.
\begin{quote}
\begin{quote}\begin{description}
\item[{param mzML}] \leavevmode
The original mzML files

\item[{param avroFile}] \leavevmode
The avro file the spectrumData is written to

\item[{param typeDict}] \leavevmode
Needed to find data types

\item[{param avro\_schema}] \leavevmode
Schema of the whole mzML

\item[{param metaDataFile}] \leavevmode
File the to which the metaData is written to

\item[{param spectrumAvsc}] \leavevmode
Schema for writing the spectrum Data

\item[{param specialXML}] \leavevmode
True if mzML is indexed, False if not

\item[{param indexJSON}] \leavevmode
Filename of the index file

\end{description}\end{quote}
\end{quote}

\end{fulllineitems}

\index{mzAvroWriter() (mzMLWriter.mzMLWriter method)}

\begin{fulllineitems}
\phantomsection\label{mzMLWriter:mzMLWriter.mzMLWriter.mzAvroWriter}\pysiglinewithargsret{\bfcode{mzAvroWriter}}{\emph{XML}, \emph{writeDict}}{}
mzAvroWriter takes a spectrum XML as well as additional data stored in a dict that are appended to the file.
Custom Datafields can be added to the schema using the appendCustomField method in the SchemaBuilder class.
Usage:

\begin{Verbatim}[commandchars=\\\{\}]
\PYG{g+gp}{\PYGZgt{}\PYGZgt{}\PYGZgt{} }\PYG{k+kn}{import} \PYG{n+nn}{pymzml}
\PYG{g+gp}{\PYGZgt{}\PYGZgt{}\PYGZgt{} }\PYG{k+kn}{import} \PYG{n+nn}{pymzavro}
\PYG{g+gp}{\PYGZgt{}\PYGZgt{}\PYGZgt{} }\PYG{n}{mzAvroWriter} \PYG{o}{=} \PYG{n}{pymzavro}\PYG{o}{.}\PYG{n}{mzMLWriter}\PYG{o}{.}\PYG{n}{mzMLWriter}\PYG{p}{(}\PYG{p}{)}
\PYG{g+gp}{\PYGZgt{}\PYGZgt{}\PYGZgt{} }\PYG{n}{mzML} \PYG{o}{=} \PYG{n+nb}{open}\PYG{p}{(}\PYG{l+s}{\PYGZdq{}}\PYG{l+s}{F04.mzML}\PYG{l+s}{\PYGZdq{}}\PYG{p}{)}
\PYG{g+gp}{\PYGZgt{}\PYGZgt{}\PYGZgt{} }\PYG{n}{spectrumFile} \PYG{o}{=} \PYG{n+nb}{open}\PYG{p}{(}\PYG{l+s}{\PYGZdq{}}\PYG{l+s}{F04.avro}\PYG{l+s}{\PYGZdq{}}\PYG{p}{,} \PYG{l+s}{\PYGZdq{}}\PYG{l+s}{wb}\PYG{l+s}{\PYGZdq{}}\PYG{p}{)}
\PYG{g+gp}{\PYGZgt{}\PYGZgt{}\PYGZgt{} }\PYG{n}{mzAvroWriter}\PYG{o}{.}\PYG{n}{init\PYGZus{}file}\PYG{p}{(}\PYG{n}{mzML}\PYG{p}{,} \PYG{n}{spectrumFile}\PYG{p}{)}
\PYG{g+gp}{\PYGZgt{}\PYGZgt{}\PYGZgt{} }\PYG{n}{mzAvroWriter}\PYG{o}{.}\PYG{n}{initmzAvroWriter}\PYG{p}{(}\PYG{p}{)}
\PYG{g+gp}{\PYGZgt{}\PYGZgt{}\PYGZgt{} }\PYG{n}{mzMLReader} \PYG{o}{=} \PYG{n}{pymzml}\PYG{o}{.}\PYG{n}{run}\PYG{o}{.}\PYG{n}{Reader}\PYG{p}{(}\PYG{l+s}{\PYGZdq{}}\PYG{l+s}{F04.mzML}\PYG{l+s}{\PYGZdq{}}\PYG{p}{)}
\PYG{g+gp}{\PYGZgt{}\PYGZgt{}\PYGZgt{} }\PYG{k}{for} \PYG{n}{spectrum} \PYG{o+ow}{in} \PYG{n}{reader}\PYG{p}{:}
\PYG{g+gp}{\PYGZgt{}\PYGZgt{}\PYGZgt{} }    \PYG{n}{xml} \PYG{o}{=} \PYG{n}{spectrum}\PYG{o}{.}\PYG{n}{xmlTreeIterFree}
\PYG{g+gp}{\PYGZgt{}\PYGZgt{}\PYGZgt{} }    \PYG{n}{mzAvroWriter}\PYG{o}{.}\PYG{n}{mzAvroWriter}\PYG{p}{(}\PYG{n}{xml}\PYG{p}{,} \PYG{p}{\PYGZob{}}\PYG{l+s}{\PYGZdq{}}\PYG{l+s}{mzArray}\PYG{l+s}{\PYGZdq{}} \PYG{p}{:} \PYG{n}{spectrum}\PYG{o}{.}\PYG{n}{mz}\PYG{p}{,} \PYG{l+s}{\PYGZdq{}}\PYG{l+s}{intensityArray}\PYG{l+s}{\PYGZdq{}} \PYG{p}{:} \PYG{n}{spectrum}\PYG{o}{.}\PYG{n}{i}\PYG{p}{\PYGZcb{}}\PYG{p}{)}
\end{Verbatim}
\begin{quote}\begin{description}
\item[{Parameters}] \leavevmode\begin{itemize}
\item {} 
\textbf{\texttt{XML}} -- XML of one spectrum from spectrumlist

\item {} 
\textbf{\texttt{writeDict}} -- dict with additional data e.g. decoded data arrays

\end{itemize}

\end{description}\end{quote}

\end{fulllineitems}

\index{simplemzAvroWriter() (mzMLWriter.mzMLWriter method)}

\begin{fulllineitems}
\phantomsection\label{mzMLWriter:mzMLWriter.mzMLWriter.simplemzAvroWriter}\pysiglinewithargsret{\bfcode{simplemzAvroWriter}}{\emph{indexWrite='advanced'}}{}
simplemzAvroWriter is used to write spectrum data to an avro file, the information about the files are provided using
init\_file method. The indexWrite option is used to write an index based on the byte offsets of the fileto enable
random access. The spectrum data is represented by a record that stores all the information that are stored in a
spectrum child in the original mzML.
Usage:

\begin{Verbatim}[commandchars=\\\{\}]
\PYG{g+gp}{\PYGZgt{}\PYGZgt{}\PYGZgt{} }\PYG{k+kn}{import} \PYG{n+nn}{pymzavro}
\end{Verbatim}
\begin{description}
\item[{Needed files are opened}] \leavevmode
\begin{Verbatim}[commandchars=\\\{\}]
\PYG{g+gp}{\PYGZgt{}\PYGZgt{}\PYGZgt{} }\PYG{n}{mzML} \PYG{o}{=} \PYG{n+nb}{open}\PYG{p}{(}\PYG{l+s}{\PYGZdq{}}\PYG{l+s}{/home/marius/data/F/mzML/F04.mzML}\PYG{l+s}{\PYGZdq{}}\PYG{p}{,} \PYG{l+s}{\PYGZdq{}}\PYG{l+s}{r}\PYG{l+s}{\PYGZdq{}}\PYG{p}{)}
\PYG{g+gp}{\PYGZgt{}\PYGZgt{}\PYGZgt{} }\PYG{n}{spectrumFile} \PYG{o}{=} \PYG{n+nb}{open}\PYG{p}{(}\PYG{l+s}{\PYGZdq{}}\PYG{l+s}{F04\PYGZus{}deflate.avro}\PYG{l+s}{\PYGZdq{}}\PYG{p}{,} \PYG{l+s}{\PYGZdq{}}\PYG{l+s}{wb}\PYG{l+s}{\PYGZdq{}}\PYG{p}{)}
\PYG{g+gp}{\PYGZgt{}\PYGZgt{}\PYGZgt{} }\PYG{n}{spectrumSchema} \PYG{o}{=} \PYG{n+nb}{open}\PYG{p}{(}\PYG{l+s}{\PYGZdq{}}\PYG{l+s}{spectrum.avsc}\PYG{l+s}{\PYGZdq{}}\PYG{p}{,}\PYG{l+s}{\PYGZdq{}}\PYG{l+s}{r}\PYG{l+s}{\PYGZdq{}}\PYG{p}{)}
\end{Verbatim}

\item[{Initialize writer class}] \leavevmode
\begin{Verbatim}[commandchars=\\\{\}]
\PYG{g+gp}{\PYGZgt{}\PYGZgt{}\PYGZgt{} }\PYG{n}{writer} \PYG{o}{=} \PYG{n}{pymzavro}\PYG{o}{.}\PYG{n}{mzMLWriter}\PYG{o}{.}\PYG{n}{mzMLWriter}\PYG{p}{(}\PYG{p}{)}
\PYG{g+gp}{\PYGZgt{}\PYGZgt{}\PYGZgt{} }\PYG{n}{writer}\PYG{o}{.}\PYG{n}{init\PYGZus{}file}\PYG{p}{(}\PYG{n}{mzML}\PYG{p}{,} \PYG{n}{spectrumFile}\PYG{p}{,} \PYG{n}{spectrumAvsc}\PYG{o}{=}\PYG{n}{spectrumSchema}\PYG{p}{)}
\end{Verbatim}

\item[{start simple conversion}] \leavevmode
\begin{Verbatim}[commandchars=\\\{\}]
\PYG{g+gp}{\PYGZgt{}\PYGZgt{}\PYGZgt{} }\PYG{n}{writer}\PYG{o}{.}\PYG{n}{simplemzAvroWriter}\PYG{p}{(}\PYG{p}{)}
\end{Verbatim}
\begin{quote}\begin{description}
\item[{param indexWrite}] \leavevmode
Used to specify the indexing type that should be used, currently only advanced available

\end{description}\end{quote}

\end{description}

\end{fulllineitems}

\index{writeOffsetToJson() (mzMLWriter.mzMLWriter method)}

\begin{fulllineitems}
\phantomsection\label{mzMLWriter:mzMLWriter.mzMLWriter.writeOffsetToJson}\pysiglinewithargsret{\bfcode{writeOffsetToJson}}{}{}
Used to write the created index to a dict.

\end{fulllineitems}

\index{writemzAvroMeta() (mzMLWriter.mzMLWriter method)}

\begin{fulllineitems}
\phantomsection\label{mzMLWriter:mzMLWriter.mzMLWriter.writemzAvroMeta}\pysiglinewithargsret{\bfcode{writemzAvroMeta}}{}{}
Writes the metaData to the specified meta data avro file. Metadata are defined as all data from mzML that is
not stored under a spectrum child.

\end{fulllineitems}


\end{fulllineitems}



\chapter{reader}
\label{reader::doc}\label{reader:module-reader}\label{reader:reader}\index{reader (module)}\index{PymzAvroReader (class in reader)}

\begin{fulllineitems}
\phantomsection\label{reader:reader.PymzAvroReader}\pysiglinewithargsret{\strong{class }\code{reader.}\bfcode{PymzAvroReader}}{\emph{avromz}, \emph{metaDataFile=None}, \emph{readerType=True}, \emph{indexFile='index.json'}}{}
The pymzAvroReader class is used to give basic access to the stored file data. Its iterable and each iteration returns
a avroSpectrum object that stores the information of the current spectrum and enables basic access to the data.
The pymzAvroReader also enables random access to spectra using the rndSeek method.
To initialize the reader, a spectrum file has to be handed over. Optionally a metadata file can be handed over.
If a metadatafile is available, the metadata are added to spectrum Object.
Example:
\#iterating across a whole avromz file and printing the mzArray of each spectrum

\begin{Verbatim}[commandchars=\\\{\}]
\PYG{g+gp}{\PYGZgt{}\PYGZgt{}\PYGZgt{} }\PYG{k+kn}{import} \PYG{n+nn}{pymzavro}
\PYG{g+gp}{\PYGZgt{}\PYGZgt{}\PYGZgt{} }\PYG{n}{spectrumAvroFile} \PYG{o}{=} \PYG{n+nb}{open}\PYG{p}{(}\PYG{l+s}{\PYGZdq{}}\PYG{l+s}{file}\PYG{l+s}{\PYGZdq{}}\PYG{p}{)}
\PYG{g+gp}{\PYGZgt{}\PYGZgt{}\PYGZgt{} }\PYG{n}{run} \PYG{o}{=} \PYG{n}{pymzAvroReader}\PYG{p}{(}\PYG{n}{spectrumAvroFile}\PYG{p}{)}
\PYG{g+gp}{\PYGZgt{}\PYGZgt{}\PYGZgt{} }\PYG{k}{for} \PYG{n}{spectrum} \PYG{o+ow}{in} \PYG{n}{run}\PYG{p}{:}
\PYG{g+gp}{\PYGZgt{}\PYGZgt{}\PYGZgt{} }    \PYG{n}{mzArray} \PYG{o}{=} \PYG{n+nb+bp}{self}\PYG{o}{.}\PYG{n}{getmzArray}\PYG{p}{(}\PYG{p}{)}
\PYG{g+gp}{\PYGZgt{}\PYGZgt{}\PYGZgt{} }    \PYG{k}{print}\PYG{p}{(}\PYG{n}{mzArray}\PYG{p}{)}
\end{Verbatim}
\begin{description}
\item[{\#seek for a spectrum with a specific index}] \leavevmode
\begin{Verbatim}[commandchars=\\\{\}]
\PYG{g+gp}{\PYGZgt{}\PYGZgt{}\PYGZgt{} }\PYG{k+kn}{import} \PYG{n+nn}{pymzavro}
\PYG{g+gp}{\PYGZgt{}\PYGZgt{}\PYGZgt{} }\PYG{n}{spectrumAvroFile} \PYG{o}{=} \PYG{n+nb}{open}\PYG{p}{(}\PYG{l+s}{\PYGZdq{}}\PYG{l+s}{file}\PYG{l+s}{\PYGZdq{}}\PYG{p}{)}
\PYG{g+gp}{\PYGZgt{}\PYGZgt{}\PYGZgt{} }\PYG{n}{indexFile} \PYG{o}{=} \PYG{n}{indexFilePath}
\PYG{g+gp}{\PYGZgt{}\PYGZgt{}\PYGZgt{} }\PYG{n}{run} \PYG{o}{=} \PYG{n}{pymzAvroReader}\PYG{p}{(}\PYG{n}{spectrumAvroFile}\PYG{p}{)}
\PYG{g+gp}{\PYGZgt{}\PYGZgt{}\PYGZgt{} }\PYG{n}{run}\PYG{o}{.}\PYG{n}{rndSeek}\PYG{p}{(}\PYG{n}{index}\PYG{p}{,} \PYG{n}{indexFile}\PYG{p}{)}
\end{Verbatim}

\end{description}

More methods that provide provide data access please refer to the avroSpectrum class.
\index{rndSeek() (reader.PymzAvroReader method)}

\begin{fulllineitems}
\phantomsection\label{reader:reader.PymzAvroReader.rndSeek}\pysiglinewithargsret{\bfcode{rndSeek}}{\emph{index}, \emph{avromz}}{}
Enables random seek
:param index: number of the index
:param avromz: path to the mzavroFile
:return: dictionary representation of th current spectrum

\end{fulllineitems}


\end{fulllineitems}



\chapter{AvroSpectrumClass}
\label{avroSpectrum:avrospectrumclass}\label{avroSpectrum:module-avroSpectrum}\label{avroSpectrum::doc}\index{avroSpectrum (module)}\index{avroSpectrum (class in avroSpectrum)}

\begin{fulllineitems}
\phantomsection\label{avroSpectrum:avroSpectrum.avroSpectrum}\pysigline{\strong{class }\code{avroSpectrum.}\bfcode{avroSpectrum}}
Stores basic information about current spectra and and makes it accessible
\index{addcvParamLocList() (avroSpectrum.avroSpectrum method)}

\begin{fulllineitems}
\phantomsection\label{avroSpectrum:avroSpectrum.avroSpectrum.addcvParamLocList}\pysiglinewithargsret{\bfcode{addcvParamLocList}}{\emph{addList}}{}~\begin{description}
\item[{Used to add a list as a path to a cvParam to get cvParam data from there}] \leavevmode\begin{description}
\item[{:param}] \leavevmode
addList: list with the path to the access target, currently only additional cvParams are enabled

\end{description}

\end{description}

\end{fulllineitems}

\index{getByAccession() (avroSpectrum.avroSpectrum method)}

\begin{fulllineitems}
\phantomsection\label{avroSpectrum:avroSpectrum.avroSpectrum.getByAccession}\pysiglinewithargsret{\bfcode{getByAccession}}{\emph{accession}}{}~\begin{description}
\item[{Returns the value for the accession (obo Tag) from MSDict}] \leavevmode\begin{quote}\begin{description}
\item[{param accession}] \leavevmode
OBO Accession number, e.g.: ``MS:1000014''

\item[{return}] \leavevmode
value from cvParam for current accession

\end{description}\end{quote}

\end{description}

\end{fulllineitems}

\index{getIntensityArray() (avroSpectrum.avroSpectrum method)}

\begin{fulllineitems}
\phantomsection\label{avroSpectrum:avroSpectrum.avroSpectrum.getIntensityArray}\pysiglinewithargsret{\bfcode{getIntensityArray}}{}{}
Returns a list of intensity values of the current spectum, the values come from pymzML.spectrum.i,
so if they were decoded in the original
mzML file, they are decoded by mzML and then written to the avro file.
:rtype: list
:return: List of intensity values from the current spectrum

\end{fulllineitems}

\index{getMSDict() (avroSpectrum.avroSpectrum method)}

\begin{fulllineitems}
\phantomsection\label{avroSpectrum:avroSpectrum.avroSpectrum.getMSDict}\pysiglinewithargsret{\bfcode{getMSDict}}{}{}~\begin{description}
\item[{Used to return a dict with all the cvParam informations extracted}] \leavevmode\begin{quote}\begin{description}
\item[{rtype}] \leavevmode
dict

\item[{return}] \leavevmode
Returns the  dict with all extracted cvParam informations

\end{description}\end{quote}

\end{description}

\end{fulllineitems}

\index{getSpectrum() (avroSpectrum.avroSpectrum method)}

\begin{fulllineitems}
\phantomsection\label{avroSpectrum:avroSpectrum.avroSpectrum.getSpectrum}\pysiglinewithargsret{\bfcode{getSpectrum}}{}{}
Returns a representation of the currently loaded spectrum, either object or dictionary, depending on datasource
(iterating returns object, seeking returns dictionary)
:return: self.spectrum

\end{fulllineitems}

\index{getmzArray() (avroSpectrum.avroSpectrum method)}

\begin{fulllineitems}
\phantomsection\label{avroSpectrum:avroSpectrum.avroSpectrum.getmzArray}\pysiglinewithargsret{\bfcode{getmzArray}}{}{}
Returns a list of mz values, thevalues come from pymzML.spectrum.mz, so if they were decoded in the original
mzML file, they are decoded by mzML and then written to the avro file.
:rtype: list
:return: List of mzValues from the current spectrum

\end{fulllineitems}

\index{iterOvercvParam() (avroSpectrum.avroSpectrum method)}

\begin{fulllineitems}
\phantomsection\label{avroSpectrum:avroSpectrum.avroSpectrum.iterOvercvParam}\pysiglinewithargsret{\bfcode{iterOvercvParam}}{}{}
Reads cvParams specified in cvParamLocList and stores them with their accession number in a dictionary called
MSDict

\end{fulllineitems}

\index{setData() (avroSpectrum.avroSpectrum method)}

\begin{fulllineitems}
\phantomsection\label{avroSpectrum:avroSpectrum.avroSpectrum.setData}\pysiglinewithargsret{\bfcode{setData}}{\emph{avroSpectrum}}{}~\begin{description}
\item[{Loads the current spectrum data to the avroSpectrum class}] \leavevmode\begin{quote}\begin{description}
\item[{param avroSpectrum}] \leavevmode
Avro spectrum class

\end{description}\end{quote}

\end{description}

\end{fulllineitems}

\index{setFromDict() (avroSpectrum.avroSpectrum method)}

\begin{fulllineitems}
\phantomsection\label{avroSpectrum:avroSpectrum.avroSpectrum.setFromDict}\pysiglinewithargsret{\bfcode{setFromDict}}{\emph{currentDict}}{}
Function to build MSDict from a dictionary (e.g. for seeking)
:param currentDict: Dict that represents the data of the spectrum
:return:

\end{fulllineitems}

\index{setMSPropDict() (avroSpectrum.avroSpectrum method)}

\begin{fulllineitems}
\phantomsection\label{avroSpectrum:avroSpectrum.avroSpectrum.setMSPropDict}\pysiglinewithargsret{\bfcode{setMSPropDict}}{}{}~\begin{description}
\item[{Sets minimum OBO list derived from pymzML minimum OBO, used to know wich attribute to get from cvParam}] \leavevmode\begin{quote}\begin{description}
\item[{return}] \leavevmode
\end{description}\end{quote}

\end{description}

\end{fulllineitems}


\end{fulllineitems}



\chapter{Indices and tables}
\label{index:indices-and-tables}\begin{itemize}
\item {} 
\DUspan{xref,std,std-ref}{genindex}

\item {} 
\DUspan{xref,std,std-ref}{modindex}

\item {} 
\DUspan{xref,std,std-ref}{search}

\end{itemize}


\renewcommand{\indexname}{Python Module Index}
\begin{theindex}
\def\bigletter#1{{\Large\sffamily#1}\nopagebreak\vspace{1mm}}
\bigletter{a}
\item {\texttt{avroSpectrum}}, \pageref{avroSpectrum:module-avroSpectrum}
\indexspace
\bigletter{m}
\item {\texttt{mzMLWriter}}, \pageref{mzMLWriter:module-mzMLWriter}
\indexspace
\bigletter{r}
\item {\texttt{reader}}, \pageref{reader:module-reader}
\end{theindex}

\renewcommand{\indexname}{Index}
\printindex
\end{document}
